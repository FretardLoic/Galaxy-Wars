\documentclass[a4paper]{report}
\usepackage[utf8]{inputenc}
\usepackage[T1]{fontenc}
\usepackage[francais]{babel}

\title{Projet Langages Web}
\author{\textsc{Hardouin} Paul\\\textsc{Frétard} Loïc}
\date{2 avril 2017}
\begin{document}
\maketitle
\chapter*{Fonctionnalités}
\section*{Page d'accueil}

La page d'accueil (index.php) permet de paramétrer plusieurs chose avant de lancer une partie:
\begin{itemize}
 \item le nombre de joueurs humains
 \item le nombre d'intelligences artificielles (IA)
 \item le nombre de planètes
\end{itemize}

Le bouton valider nous amène vers la page name\_enter.php qui permet d'entrer
les noms des joueurs humains.

\section*{Déroulement d'une partie}

Tous les joueurs jouent, chacun leur tour.
Lorsque c'est au tour d'un humain de jouer, l'utilisateur peut s'il le souhaite,
lancer une flotte depuis une de ses planètes vers une autre. Un envoi de flotte
se déroule de la façon suivante: un premier clique sur une première planète appartenant au
joueur actif suivi d'un deuxième clic sur n'importe quelle planète, déclenche une
fenêtre demandant à l'utilisateur combien d'individu la flotte doit emporter. Bien sur,
le joueur peut envoyer plusieurs flottes s'il le souhaite, ou pas du tout.
Lorsqu'il a fini de lancer ses flottes, le joueur peut finir son tour en cliquant sur le bouton
``fin du tour'' et le prochain joueur peut comencer son tour. Si le prochain joueur est une IA,
Les déplacements de flottes ainsi que la croissance des populations sur les
planètes se font lors du tour du joueur associé à ces éléments.
Lorsqu'un joueur ne dispose plus d'aucune planète ni d'aucune flotte,
alors ce joueur perd la partie et est retiré de la liste des joueurs
La partie se fini lorsqu'il ne reste plus qu'un joueur.

\chapter*{Fonctionnement}
\section*{Biomass}

La classe Biomass est représente une population. 
Elle est définie par :
\begin{itemize}
 \item une quantité d'individu
 \item une position
 \item un rayon
 \item une faction (un joueur)
 \item un contexte pour pouvoir s'afficher
\end{itemize}

\section*{Planet}

La classe Planet est une sous-classe de Biomass qui possédent une croissance (et une image) qui modélise une planète. Une planète est neutre si elle n'a pas de joueur.

\section*{Fleet}

Fleet est également une sous-classe de Biomass mais qui représente une flotte en mouvement. Une Fleet posséde une destination qui est une instance de Planet.

\section*{Player}

Player représente un joueur et possède :
\begin{itemize}
 \item un nom
 \item un ensemble de Biomass
 \item une couleur
\end{itemize}
Un joueur a perdu lorsqu'il n'a plus de Biomass.

\section*{IAPlayer}

La classe IAPlayer est une sous-classe de Player pour lequel une fonction playTurn a été définie pour lui permettre de jouer un tour sans intervention extérieur. Pour pouvoir joué, ce joueur a besoin de la Galaxy à laquelle il appartient.

\section*{Galaxy}

La classe Galaxy permet de modéliser et d'afficher le jeu. Une galaxy est composé :
\begin{itemize}
 \item d'une liste de joueurs
 \item d'une liste de planètes
 \item d'une image (pour le fond)
 \item d'un contexte (pour s'afficher dans un canvas)
 \item d'une fonction qui s’exécute quand la partie est finie
\end{itemize}
Ainsi, la Galaxy peut organiser le déroulement de la partie en passant d'un joueur à l'autre et ordonne à chaque tour aux Planet et au Fleet s'afficher.
C'est également elle qui lance la fonction d'éxecution des IAPlayer.

De plus, il y a un générateur de Galaxy qui prend en argument :
\begin{itemize}
 \item une liste de noms
 \item un nombre d’IA
 \item un nombre de planètes
 \item une image (pour le fond)
 \item un contexte (pour s'afficher dans un canvas)
 \item une fonction de fin de partie.
\end{itemize}

\end{document}
