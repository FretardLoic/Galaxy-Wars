\documentclass[a4paper]{report}
\usepackage[utf8]{inputenc}
\usepackage[T1]{fontenc}
\usepackage[francais]{babel}

\title{Projet Langages Web}
\author{\textsc{Hardouin} Paul\\\textsc{Frétard} Loïc}
\date{2 avril 2017}
\begin{document}
\maketitle
\chapter*{Fonctionnalités}
\section*{Page d'accueil}

La page d'accueil (index.php) permet de paramétrer plusieurs chose avant de lancer une partie:
\begin{itemize}
 \item le nombre de joueurs humains
 \item le nombre d'intelligences artificielles (IA)
 \item le nombre de planètes
\end{itemize}

Le bouton valider nous amène vers la page name\_enter.php qui permet d'entrer
les noms des joueurs humains.

\section*{Déroulement d'une partie}

Tous les joueurs jouent, chacun leur tour.
Lorsque c'est au tour d'un humain de jouer, l'utilisateur peut s'il le souhaite,
lancer une flotte depuis une de ses planètes vers une autre. Un envoi de flotte
se déroule de la façon suivante: un premier clique sur une première planète appartenant au
joueur actif suivi d'un deuxième clic sur n'importe quelle planète, déclenche une
fenêtre demandant à l'utilisateur combien d'individu la flotte doit emporter. Bien sur,
le joueur peut envoyer plusieurs flottes s'il le souhaite, ou pas du tout.
Lorsqu'il a fini de lancer ses flottes, le joueur peut finir son tour en cliquant sur le bouton
``fin du tour'' et le prochain joueur peut comencer son tour. Si le prochain joueur est une IA,
Les déplacements de flottes ainsi que la croissance des populations sur les
planètes se font lors du tour du joueur associé à ces éléments.
Lorsqu'un joueur ne dispose plus d'aucune planète ni d'aucune flotte,
alors ce joueur perd la partie et est retiré de la liste des joueurs
La partie se fini lorsqu'il ne reste plus qu'un joueur.

\end{document}
